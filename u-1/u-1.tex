\documentclass[a4paper,ngerman]{scrartcl}

%Compiler
\usepackage{ifxetex}
\usepackage{ifluatex}
\ifxetex
  \usepackage{fontspec,xunicode}
  \catcode`\ß=13
  \defß{\ss}
\else\ifluatex
  \usepackage{fontspec,xunicode}
\else
  \usepackage[utf8]{inputenc}
\fi\fi
% /Compiler

\usepackage{fullpage}
\usepackage{minted}
\usepackage{amsmath}

\usepackage[ngerman]{babel}

\begin{document}
\noindent WS 13/14\hfill \textbf{Computerphysik}\hfill Carlos Martín Nieto\\
Blatt 1 | 27.10.2013 \hfill Tran Tu\\

\noindent\hrulefill

\section*{Aufgabe 1.1}

\paragraph{1.1.1}

\inputminted[mathescape,linenos]{python}{aufgabe1.py}

\texttt{\% python aufgabe1.py}

\[
\begin{array}{|l|l|}\hline
  \alpha & \epsilon\\\hline
  1       &1.11022\cdot 10^{-16}\\\hline
  1.1\cdot 10^{-5} &8.47033\cdot 10^{-22}\\\hline
  1\cdot 10^{-30}   &8.75812\cdot 10^{-47}\\\hline
  5\cdot 10^{12}   &0.000488281\\\hline
\end{array}
\]
\paragraph{1.1.2}

$\epsilon$ ist der Wert, was addiert zu $\alpha$ keine Änderung des Wertes erzeugt. Das nennt man Machinengenauigkeit.

\paragraph{1.1.3}

Ja, aus dem Standard lässt sich die Machinengenauigkeit bestimmen.

\[
\varepsilon_{\text{max}} = 2^{-n} = 2^{-23} = 1.1920928955078125\cdot 10^{-7}
\]

\end{document}
