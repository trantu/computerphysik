\documentclass[a4paper,ngerman]{scrbook}

\usepackage{amsmath}
\usepackage{amssymb}
\usepackage[ngerman]{babel}
\usepackage{hyperref}

%Compiler
\usepackage{ifxetex}
\usepackage{ifluatex}
\ifxetex
  \usepackage{fontspec,xunicode}
  \catcode`\ß=13
  \defß{\ss}
\else\ifluatex
  \usepackage{fontspec,xunicode}
\else
  \usepackage[utf8]{inputenc}
\fi\fi
% /Compiler

\newcommand{\bO}{\ensuremath{\mathcal{O}}}%

\title{Computergestützte Methoden der exakten Naturwissenschaften }
\subtitle{Computerphysik}
\date{Wintersemester 2013/2014}
\author{Carlos Martín Nieto}
\begin{document}
\maketitle
\chapter{Fehler}

Ziel der Naturwissenschaften: Beschreibung der natur durch (einfache) Gleichungen und deren Lösungen

\paragraph{Problem}

Gleichungen typischerweise nicht lösbar mit Papier und Bleistift.
\begin{itemize}
\item[Lösung 1] Gleichungen vereinfachen $\hat{=}$ Näherung/Approximation
\item[Lösung 2] Numerische Lösung von Gleichungen $\implies$ \boxed{\text{diese Vorlesung}}
\end{itemize}

in Naturwissenschaften wichtig: Genauigkeit (den Fehler) von numerischen Rechenergebnissen beurteilen! Es gibt verschiedene Fehlerquellen
\begin{enumerate}
\item Eingabefehler durch Ungenauigkeiten in den Eingabedaten.
\item Näherungsfehler wenn statt exakten mathematische Ausdrücke vereinfachte Ausdrücke benutzt werden.
\item Modelfehler durch vereinfachte physikalische Modelle.
\item Rundungsfehler durch die numerische Darstellung von Zahlen und der damit verbundenden endlichen Genauigkeiten.
\end{enumerate}

\section{Näherungsfehler}

Viele mathemiatische Ausdrúcke, die in der Physik auftreten sind in der exakten Formulierung nicht oder sehr aufwendig zu berechnen $\to$ Approximation $\to$ Näherungsfehler. Häufig: Funktionen definiert durch unendliche Reihen.

\paragraph{Bsp 1}
Exponenzialfunktionen

Die Funktion ist $e^x = \displaystyle\sum^\infty_{n=0} \frac{x^n}{n!}$. Die Näherung $e^x = \displaystyle\sum^N_{n=0} \frac{x^n}{n!}$

$\displaystyle\frac{df(x)}{dx} = F(x)$ werden durch e-Funktion gelöst.

\paragraph{Bsp 2}
Lösung einer Differenzialgleichung im Kontinuum wird ersetzt durch Lösung der diskretisierten Gleichung.

Ausgangs-DG: $\displaystyle\frac{d}{dx}f(x) = a\cdot f(x)$, Lösung $f(x) = e^{ax}$.

Lösung durch Beschränkung auf diskrete Gitterpunkte. $x_i$ mit $x_{i+1} - x_i = \Delta x \to$

\[
f(x) = \frac{f(x_{i+1}) - f(x_i)}{x_{i+1} - x_i} = a\cdot \frac{f(x_{i+1}) - f(x_i)}{2}
\]

Verbesserung der Näherung durch Verfeinerung der Diskretisierung, also $\Delta x \to 0$.

Nachteil: mehr Rechenoperationen und damit mehr Rechenzeit, außerdem mehr Rundingsfehler (mehr in \autoref{sec:rundungsfehler}).

Ziel der Numerik: optimaler Kompromiss zwischen Fehler und Rechenzeit finden.
\section{Modellfehler}
\label{sec:modellfehler}

\paragraph{Beispiel}
Planetenbewegung

Erster Keppler Gesetzt: Planeten bewegen sich auf elyptischen Bahnen, in einem Brennpunkt steht die Sonne.

Neutonische Bewegungsgleichung
\[
\vec{F} = m\cdot \vec{a} = m\cdot \frac{d^2\vec{r}(t)}{dt^2} = -G \frac{Mm\vec{r}}{|r|^3}
\]

\paragraph{Modelnäherungen}

\begin{enumerate}
\item Sonnenmass $M \gg$ Planetenmasse $m$ (leicht korrigierbar)
\item Reibungskraft $\vec{F}_R = -\gamma \frac{d\vec{r}(t)}{dt}$ vernachlässigt. OK für Planeten, wichtig für kleine Objekte.
\item Gravitationsgesetzt in der einfachen Form gilt nur für Kugeln!
\item Relativistische Effekte $\to$ Merkur-Perikeldrehung.
\item Mehrkörperproblem $r_i(t)$, $i=1,\dots,N$
  \[
  m_i \frac{d^2r_i(t)}{dt^2} = \vec{F_i} = -\sum_{j\neq i} G m_i m_j \frac{\vec{r_i}(t) - \vec{r_j}(t)}{|v_i(t) - r_j(t)|^3}
  \]
\end{enumerate}

Gleichung für $N>2$ wechselwirkende Massen kann leicht hingeschrieben werden (durch Summatia der Knüpfe). Lösung aber nur nummerisch möglich! Annahme: $N=3$ Dreikörperproblem als 3-Stöße!

\end{document}
