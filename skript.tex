\documentclass[a4paper,ngerman]{scrbook}

\usepackage{amsmath}
\usepackage{amssymb}
\usepackage[ngerman]{babel}
\usepackage{hyperref}

%Compiler
\usepackage{ifxetex}
\usepackage{ifluatex}
\ifxetex
  \usepackage{fontspec,xunicode}
  \catcode`\ß=13
  \defß{\ss}
\else\ifluatex
  \usepackage{fontspec,xunicode}
\else
  \usepackage[utf8]{inputenc}
\fi\fi
% /Compiler

\newcommand{\bO}{\ensuremath{\mathcal{O}}}%
\newcommand{\R}{\ensuremath{\mathds{R}}}%

\title{Computergestützte Methoden der exakten Naturwissenschaften }
\subtitle{Computerphysik}
\date{Wintersemester 2013/2014}
\author{Carlos Martín Nieto}
\begin{document}
\maketitle
\chapter{Fehler}

Ziel der Naturwissenschaften: Beschreibung der natur durch (einfache) Gleichungen und deren Lösungen

\paragraph{Problem}

Gleichungen typischerweise nicht lösbar mit Papier und Bleistift.
\begin{itemize}
\item[Lösung 1] Gleichungen vereinfachen $\hat{=}$ Näherung/Approximation
\item[Lösung 2] Numerische Lösung von Gleichungen $\implies$ \boxed{\text{diese Vorlesung}}
\end{itemize}

in Naturwissenschaften wichtig: Genauigkeit (den Fehler) von numerischen Rechenergebnissen beurteilen! Es gibt verschiedene Fehlerquellen
\begin{enumerate}
\item Eingabefehler durch Ungenauigkeiten in den Eingabedaten.
\item Näherungsfehler wenn statt exakten mathematische Ausdrücke vereinfachte Ausdrücke benutzt werden.
\item Modelfehler durch vereinfachte physikalische Modelle.
\item Rundungsfehler durch die numerische Darstellung von Zahlen und der damit verbundenden endlichen Genauigkeiten.
\end{enumerate}

\section{Näherungsfehler}

Viele mathemiatische Ausdrúcke, die in der Physik auftreten sind in der exakten Formulierung nicht oder sehr aufwendig zu berechnen $\to$ Approximation $\to$ Näherungsfehler. Häufig: Funktionen definiert durch unendliche Reihen.

\paragraph{Bsp 1}
Exponenzialfunktionen

Die Funktion ist $e^x = \displaystyle\sum^\infty_{n=0} \frac{x^n}{n!}$. Die Näherung $e^x = \displaystyle\sum^N_{n=0} \frac{x^n}{n!}$

$\displaystyle\frac{df(x)}{dx} = F(x)$ werden durch e-Funktion gelöst.

\paragraph{Bsp 2}
Lösung einer Differenzialgleichung im Kontinuum wird ersetzt durch Lösung der diskretisierten Gleichung.

Ausgangs-DG: $\displaystyle\frac{d}{dx}f(x) = a\cdot f(x)$, Lösung $f(x) = e^{ax}$.

Lösung durch Beschränkung auf diskrete Gitterpunkte. $x_i$ mit $x_{i+1} - x_i = \Delta x \to$

\[
f(x) = \frac{f(x_{i+1}) - f(x_i)}{x_{i+1} - x_i} = a\cdot \frac{f(x_{i+1}) - f(x_i)}{2}
\]

Verbesserung der Näherung durch Verfeinerung der Diskretisierung, also $\Delta x \to 0$.

Nachteil: mehr Rechenoperationen und damit mehr Rechenzeit, außerdem mehr Rundingsfehler (mehr in \autoref{sec:rundungsfehler}).

Ziel der Numerik: optimaler Kompromiss zwischen Fehler und Rechenzeit finden.
\section{Modellfehler}
\label{sec:modellfehler}

\paragraph{Beispiel}
Planetenbewegung

Erster Keppler Gesetzt: Planeten bewegen sich auf elyptischen Bahnen, in einem Brennpunkt steht die Sonne.

Neutonische Bewegungsgleichung
\[
\vec{F} = m\cdot \vec{a} = m\cdot \frac{d^2\vec{r}(t)}{dt^2} = -G \frac{Mm\vec{r}}{|r|^3}
\]

\paragraph{Modelnäherungen}

\begin{enumerate}
\item Sonnenmass $M \gg$ Planetenmasse $m$ (leicht korrigierbar)
\item Reibungskraft $\vec{F}_R = -\gamma \frac{d\vec{r}(t)}{dt}$ vernachlässigt. OK für Planeten, wichtig für kleine Objekte.
\item Gravitationsgesetzt in der einfachen Form gilt nur für Kugeln!
\item Relativistische Effekte $\to$ Merkur-Perikeldrehung.
\item Mehrkörperproblem $r_i(t)$, $i=1,\dots,N$
  \[
  m_i \frac{d^2r_i(t)}{dt^2} = \vec{F_i} = -\sum_{j\neq i} G m_i m_j \frac{\vec{r_i}(t) - \vec{r_j}(t)}{|v_i(t) - r_j(t)|^3}
  \]
\end{enumerate}

Gleichung für $N>2$ wechselwirkende Massen kann leicht hingeschrieben werden (durch Summatia der Knüpfe). Lösung aber nur nummerisch möglich! Annahme: $N=3$ Dreikörperproblem als 3-Stöße!

\chapter{Blah blah}

\chapter{Lineare Gleichungssyteme}
\section{Gauß}

Kompliziertes Zeug.

\section{LR-Zerlegung (LU in en)}

Möchte man mehrere LGS mit diasselben $A$ aber anderere rechter Seite lösen so empfiehlt es sich die Elementaren Umforomungen zu merken.

Für $A =
\begin{pmatrix}
  1 & 2 & 3\\
  6 & -2 & 2\\
  2 & 1 & -4
\end{pmatrix}
$ führen wir durch

\begin{align*}
z_2 = z_2 - 6z_1 &\to L_1 =
\begin{pmatrix}
  1 & 0 & 0\\
  -6 & 1 & 0\\
  0 & 0 & 0
\end{pmatrix}\\
  z_3 = z_3 + 3z_1 &\to L_2 =
  \begin{pmatrix}
    1 & 0 & 0\\
    0 & 1 & 0\\
    3 & 0 & 1
  \end{pmatrix}\\
  z_3 = z_3 + \frac{1}{2}z_2 &\to L_3 =
  \begin{pmatrix}
    1 & 0 & 0\\
    0 & 1 & 0\\
    0 & \frac{1}{2} & 1
  \end{pmatrix}\\
L_1A &=
\begin{pmatrix}
  1 & 2 & 3\\
  0 & -14 & -16\\
  -3 & 1 & -4
\end{pmatrix}
\end{align*}

\begin{align*}
  L_2L_1 &=
  \begin{pmatrix}
    1 & 0 & 0\\
    0 & 1 & 0\\
    3 & 0 & 1
  \end{pmatrix}
  \begin{pmatrix}
    1 & 0 & 0\\
    -6 & 1 & 0\\
    0 & 0 & 1
  \end{pmatrix} =
  \begin{pmatrix}
    1 & 0 & 0\\
    -6 & 1 & 0\\
    3 & 0 & 1
  \end{pmatrix}\\
L &= L3(L_1\cdot L_1) = 
  \begin{pmatrix}
    1 & 0 & 0\\
    0 & 1 & 0\\
    0 & \frac{1}{2} & 1
  \end{pmatrix}
  \begin{pmatrix}
    1 & 0 & 0\\
    -6 & 1 & 0\\
    3 & 0 & 1
  \end{pmatrix} =
  \begin{pmatrix}
    1 & 0 & 0\\
    -6 & 1 & 0\\
    0 & \frac{1}{2} & 1
  \end{pmatrix}\\
  LA &= L_3L_2L_1A - R\\
  R &= LA =
  \begin{pmatrix}
    1 & 0 & 0\\
    -6 & 1 & 0\\
    0 & \frac{1}{2} & 1
  \end{pmatrix}
  \begin{pmatrix}
    1 & 2 & 3\\
    6 & -2 & 2\\
    2 & 1 & -4
  \end{pmatrix} =
  \begin{pmatrix}
    1 & 2 & 3\\
    0 & -14 & -16\\
    0 & 0 & -3
  \end{pmatrix}
\end{align*}

Dies ist die rechts-obere Dreiechsmatrix aus der letzten VL.

\paragraph{Bsp}

Lösung von $A\vec{x} = \vec{c}$ mit $\vec{c} =
\begin{pmatrix}
  12\\ -16\\ 2
\end{pmatrix}$

\begin{align*}
  L\vec{c} &=
  \begin{pmatrix}
    1 & 0 & 0\\
    -6 & 1 & 0\\
    0 & \frac{1}{2} & 1
  \end{pmatrix}
  \begin{pmatrix}
    12\\ -16\\ 2
  \end{pmatrix}
  =
  \begin{pmatrix}
    12 \\ -88\\ -6
  \end{pmatrix}
\end{align*}

D.h\@. wir losen $R\vec{x} = y\cdot
\begin{pmatrix}
  1 & 2 & 3 & 2\\
  0 & -14 & -16 & -88 \\
  0 & 0 & -3 & -6
\end{pmatrix}$. Rückeinsetzen
\begin{align*}
  x_3 &= \frac{-6}{-3} = 2\\
  14x_2 &= 88 - 16x_3\\
  &= 88 - 32 = 56\\
  \implies x_2 &= 4\\
  x_1 &= 12 - 2x_2 - 3x_3\\
  &= 12 - 8 - 6 = -2\\
  \implies \vec{x} =
  \begin{pmatrix}
    -2\\ 4\\ 2
  \end{pmatrix}
\end{align*}

Satz. Für jede $n \times n$-Matrix, für die der Gauß-Algorithmus durchführbar ohne Zeilenaustauschung durchführbar ist, gibt es $n\times n$-Matrizen $L$ und $R$ mit der Eigenschaften:
\begin{itemize}
\item $L$ ist eine links-untere Dreiechsmatrix mit $l_{ii} = 1$ für $i=1,\dots,n$
\item $R$ ist eine rechts-obere Dreiechsmatrix mit $r_{ii} \neq 0$ für $i=1,\dots,n$.
\item $A = L^{-1}R$ bezeichnet man als \underline{$LR$-Zerlegung} von $A$.
\end{itemize}

Es gilt: \framebox{$A\vec{x} = \vec{b} \iff L\vec{b} = y$ und $R\vec{x} = \vec{y}$}

\section{Cholesky-Zerlegung}
\label{sec:chorlesky}

\paragraph{Definition}

Eine Symmetrische $n \times n$-Matrix $A$ heißt \underline{positiv-definit}, wenn für alle $\vec{x}~\in~R^n$, $\vec{x} = 0$ gilt: $\vec{x}^{T} A \vec{x} > 0$.

\paragraph{Satz}

Für jede positiv-definierte Matrix $A$ gibe es genau eine rechts-obere Dreiechsform mit $r_{ii} > 0$ für $i=1,\dots,n$ und $A=R^{T}R$. Diese Zerlegung heißt Cholesky-Zerlegung.

Zum berechnen der Cholesky-Zerlegung geht man wie folgt vor (Pseudo-Code)


\begin{verbatim}
for i = 1,...n,
  s = a_ii - \sum_{k=1}^{i-1} r_{ki}^2 // für i=1 ist s=a_11
  if s \leq 0
    stop // $A$ ist nicht positiv definit
  else
    r_ii = \sqrt{s}
    for j = i+1,...,n // nicht-Diagonalelemente
      r_{ij} = \frac{1}{r_ii} (a_ij - \sum_{k=1}^{i-1} r_{ki}r_{kj})
    endfor
endfor
\end{verbatim}

\paragraph{Bsp}

$A =
\begin{pmatrix}
  4 & 4 & 2\\
  4 & 5 & 5\\
  2 & 5 & 26
\end{pmatrix}
$
$
  R =
  \begin{pmatrix}
    r_{11} & r_{12} & r_{13}\\
    0 & r_{22} & r_{23}\\
    0 & 0 & r_{33}
  \end{pmatrix} =
  \begin{pmatrix}
    2 & 2 & 1\\
    0 & 1 & 3\\
    0 & 0 & 4
  \end{pmatrix}
$
\begin{itemize}
\item $i=1$

    $s = a_{11} = 4 > 0 \implies r_{11} = \sqrt{4} = 2$
    \begin{itemize}
    \item $j=2$

      $r_{12} = \frac{1}{2}(4) = 2$
    \item $j=3$
      $r_{13} = \frac{1}{2}(2 - 0) = 1$
    \end{itemize}
  \item $i=2$

    $s = a_{22} - \sum_{k=1}^{i-1} r_{ki}^2 = 5 - 2^2 =1 > 0 \implies r_{22} = \sqrt{1} = 1$
\begin{itemize}
\item $j=3$

  $r_{23} = \frac{1}{1} (5 - \sum_{k=1}^1 r_{12} r_{12}) = 5 - 2 = 3$
\end{itemize}
\item $i=3$

  $s = a_{33} = a_{33} - \sum_{k=1}^{2} r_{k3}^2 = 26 - (1^2 + 3^2) = 16 > 0 \implies r_{33} = \sqrt{16} = 4$
\end{itemize}

Test:

\begin{align*}
R^TR &=
\begin{pmatrix}
  2 & 0 & 0\\
  2 & 1 & 0\\
  1 & 3 & 4
\end{pmatrix}
\begin{pmatrix}
  2 & 2 & 1\\
  0 & 1 & 3\\
  0 & 0 & 4
\end{pmatrix} =
\begin{pmatrix}
  4 & 4 & 2\\
  4 & 5 & 5\\
  2 & 5 & 26
\end{pmatrix} = A\\
\intertext{aber:}
RR^T &= 
\begin{pmatrix}
  2 & 2 & 1\\
  0 & 1 & 3\\
  0 & 0 & 4
\end{pmatrix}
\begin{pmatrix}
  2 & 0 & 0\\
  2 & 1 & 0\\
  1 & 3 & 4
\end{pmatrix} =
\begin{pmatrix}
  9 & 5 & 4\\
  5 & 10 & 12\\
  4 & 12 & 16
\end{pmatrix} \neq A
\end{align*}

\paragraph{Bemerkung}

Der numerische Aufwand des Cholesky-Verfahrens beträgt $\left(\frac{1}{6} n^3 + \frac{1}{2}n^2 - \frac{2}{3} n\right)$ Punktoperationen und $n$ Wurzelberechnungen.
Das Gauß-Verfahren benötigt $\left(\frac{n^3}{3} + \frac{n}{3}\right)$ Operationen, also für $n \geq 2$ ist Cholesky schenller!

\end{document}
